\lang{\cvsection{Projects}}{\cvsection{Projets}}
\begin{cventries}
    \cventry
        {\lang{Personnal project of an ebook reader app with easy one click translations}{Projet personnel d'une application permettant de lire des livres électroniques dans plusieurs langues }}
        {Babel Reader}
        {\lang{Ongoing}{En cours}}
        {}
        {
       \begin{cvitems}
       \item{\lang{Usage of ReactJs}{Utilisation de ReactJs}}
       \item{\lang{Usage of Firebase for the hosting and the storage of user data}{Utilisation de Firebase pour l'hébergement et les donnés des utilisateurs}}
       \item{\lang{Continuous integration and deployment using Github Actions}{Intégration continue et déploiement continue avec les Github Actions.}}
       \item{\lang{Publication of the source code on Github:}{Publication du code source sur GitHub:} \url{https://github.com/Babel-Reader/babel-reader-web}}
       \item{\lang{Deployment of a development version}{Déploiement d'une version de développement} \url{https://babel-reader-web.web.app/}}
       \end{cvitems}
        }
    \cventry
        {\lang{Second year integrator project }{Projet Integrateur de deuxième année d'université}}
        {Polydessin}
        {\lang{Winter 2020}{Hiver 2020}}
        {}
        {
       \begin{cvitems}
       \item{\lang{Creation of a webapp in typescript using Angular}{Création complète d'une webapp en typescript à l'aide de Angular}}
       \item{\lang{Backend created using node and express}{Création d'un backend avec node et express}}
       \item{\lang{Continuous integration and deployment using Bitbucket pipelines}{Intégration continue et déploiement continue avec les pipelines Bitbucket.}}
       \item{\lang{Source code available on Github:}{Publication du code source sur GitHub:} \url{https://github.com/tran-simon/LOG2990-104}}
       \item{\lang{Demo version deployed using Firebase:}{Déploiement de la version finale à l'aide de Firebase} \url{https://log2990-104.web.app/}}
       \end{cvitems}
        }
    \cventry
    {\lang{Personnal project of an android open-source app to ease the learning of math formulas}{Application Android open-source développée seul pour l'apprentissage de formules mathématiques}}
        {Math by Heart}
        {}
        {}
        {
       \begin{cvitems}
       \item{\lang{Usage of AndroidStudio and the MathJax library}{Utilisation d'Android Studio et de la librairie MathJax (LaTeX) pour afficher les formules mathématiques}}
       \item{\lang{Source code available on Github}{Publication du code source sur GitHub}: \url{https://github.com/tran-simon/MathByHeart}}
       \item{\lang{Final version released on the Playstore}{Publication de la version finale sur le PlayStore}: \url{https://play.google.com/store/apps/details?id=com.games.potato.mathbyheart&hl=en}}
       \end{cvitems}
        }
    \cventry
    {\lang{End of CEGEP project}{Projet Intégrateur de fin de CÉGEP}}
    {\lang{The mass spectrometer}{Le Spectromètre de masse}}
        {2018}
        {}
        {
            \begin{cvitems}
            \item{\lang{Creation of a scientific application in Java to simulate a mass spectrometer, a cyclotron and the behavior of particles under the effect of electromagnetic fields}{Création d'une application scientifique en Java avec Swing permettant de simuler un spectromètre de masse, un cyclotron et le comportement des particules sous l'effet de divers champs électromagnétiques}}
            \item{\lang{Usage of Apache Subversion for version control}{Utilisation de Apache Subversion pour la gestion des versions}}
            \item{\lang{Creation of a physics engine to accurately simulate particles using Euler's algorithm and Runge-Kutta's}{Création d'un moteur physique permettant des simulations fièles à la réalité à l'aide des algorithmes d'Euler et de Runge-Kutta}}
            \item{\lang{Source code available on Github}{Publication du code source final sur Github}: \url{https://github.com/tran-simon/Spectrometre}}
            \end{cvitems}
        }
    \cventry
    {\lang{Project made during the electromagnetic fields course}{Projet réalisé dans le cadre du cours de champs électromagnétiques}}
    {\lang{Radio project}{Projet Radio}}
    {\lang{Fall 2019}{Session d'automne 2019}}
	{}
	{
	    \begin{cvitems}
        \item{\lang{Creation of an AM radio}{Confection d'une radio AM permettant de capter plusieurs fréquences}}
        \item{\lang{Usage of a breadboard for the circuitry}{Réalisation du circuit sur une platine d'expérimentation}}
	    \end{cvitems}
	}
\end{cventries}
