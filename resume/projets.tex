\cvsection{Projets}
\begin{cventries}
    \cventry
        {Application Android open-source développée seul pour l'apprentissage de formules mathématiques}
        {Math by Heart}
        {}
        {}
        {
       \begin{cvitems}
       \item{Utilisation d'Android Studio et de la librairie MathJax (LaTeX) pour afficher les formules mathématiques}
       \item{Publication du code source sur GitHub: \url{https://github.com/tran-simon/MathByHeart}}
       \item{Publication de la version finale sur le PlayStore: \url{https://play.google.com/store/apps/details?id=com.games.potato.mathbyheart&hl=en}}
       \end{cvitems}
        }
    \cventry
        {Le Spectromètre de masse}
        {Projet Intégrateur de fin de CÉGEP}
        {2018}
        {}
        {
            \begin{cvitems}
            \item{Création d'une application scientifique en Java avec Swing permettant de simuler un spectromètre de masse, un cyclotron et le comportement des particules sous l'effet de divers champs électromagnétiques}
            \item{Utilisation de Apache Subversion pour la gestion des versions}
            \item{Création d'un moteur physique permettant des simulations fièles à la réalité à l'aide des algorithmes d'Euler et de Runge-Kutta}
            \item{Publication du code source final sur Github: \url{https://github.com/tran-simon/Spectrometre}}
            \end{cvitems}
        }
    \cventry
	{Projet réalisé dans le cadre du cours de champs électromagnétiques}
	{Projet Radio}
	{Session d'automne 2019}
	{}
	{
	    \begin{cvitems}
	    \item{Confection d'une radio AM permettant de capter plusieurs fréquences}
	    \item{Réalisation du circuit sur une platine d'expérimentation}
	    \end{cvitems}
	}
\end{cventries}
