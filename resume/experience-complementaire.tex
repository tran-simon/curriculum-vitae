\lang{\cvsection{Complementary Experience}}{\cvsection{Expériences complémentaires}}
\begin{cventries}
  \cventry
  {\url{https://hackatown.io/}}
    {Hackatown}
    {2019}
    {Polytechnique Montréal}
    {
      \begin{cvitems}
      \item {\lang{Development of an Android app during a hackathon: Montreal Patrol}{Développement en équipe d'une application Android avec le thème de la ville intelligente dans le cadre d'un hackathon: Montreal Patrol} \url{https://devpost.com/software/hackatown-ba7z0v}}
      \end{cvitems}
    }
  \cventry
  {\url{https://hackqc.ca}}
    {HackQC}
    {2018}
    {École de Technologie Supérieure}
    {
      \begin{cvitems}
      \item{\lang{Development of an Android app using the google maps API during a hackathon}{Développement en équipe d'une application Android sur le thème de la communauté verte et de l'environnement urbain}: Tree Catcher GO - \url{https://devpost.com/software/tree-catcher-go}}
      \end{cvitems}
	}
  \cventry
  {
\url{https://diro.umontreal.ca/departement/hackathon/hackathon-2018/}
  \url{http://hackathon.iro.umontreal.ca/2017/}
}
  {Hackathon du Département d'Informatique et de Recherche Opérationnelle de l'Université de Montréal}
  {\lang{2017 and 2018 editions}{Éditions 2017 et 2018}},
	{DIRO, Université de Montréal}
	{
	  \begin{cvitems}
	  \item{\lang{2018: Development of an AI in Java to play a game similar to Lode Runner:}{2018: Développement d'une intelligence artificielle en Java jouant à un jeu similaire à Lode Runner:} \url{https://github.com/tran-simon/NodeRunner}}
	  \item{\lang{2017: Development of an AI in Java to play against other AIs in TRON:}{2017: Développement d'une intelligence artificielle en Java capable d'affronter les IA adverses dans le jeu TRON:} \url{https://github.com/tran-simon/Tron}}
	  \end{cvitems}
	}
\end{cventries}
