\cvsection{Expériences complémentaires}
\begin{cventries}
  \cventry
  {\url{https://hackatown.io/}}
    {Hackatown}
    {2019}
    {Polytechnique Montréal}
    {
      \begin{cvitems}
      \item {Développement en équipe d'une application Android avec le thème de la ville intelligente dans le cadre d'une compétition informatique d'une durée de 24h}
      \item{Utilisation de l'API Google Maps, d'Android Studio, de Github}
        \item {Lien vers notre projet: Montreal Patrol - \url{https://devpost.com/software/hackatown-ba7z0v}}
      \end{cvitems}
    }
  \cventry
  {\url{https://hackqc.ca}}
    {HackQC}
    {2019}
    {École de Technologie Supérieure}
    {
      \begin{cvitems}
      \item{Développement en équipe d'une application Android sur le thème de la communauté verte et de l'environnement urbain}
      \item{Utilisation des données ouvertes du gouvernement du Québec, de l'API Google Maps, d'Android Studio} 
        \item{Lien vers notre projet: Tree Catcher GO - \url{https://devpost.com/software/tree-catcher-go}}
      \end{cvitems}
    }
  \cventry
  {\url{https://www.agorize.com/fr/challenges/radio-canada-2018}}
  {Hackathon 2018 de Radio-Canada}
    {2018}
    {Radio-Canada, Montréal}
    {
      \begin{cvitems}
      \item{Compétition informatique en équipe où nous avons développé une application web permettant d'évaluer la qualité d'un titre d'article à l'aide de l'apprentisage machine}
      \item{Programmation d'un backend en Java permettant de transmettre les données de Radio-Canada vers notre base de donnée Azure}
      \item{Utilisation des services Azure (Base de donnée, machine virtuelle, machine learning studio), de la base de données de Radio-Canada}
      \end{cvitems}
    }
    \cventry
  {\url{https://diro.umontreal.ca/departement/hackathon/hackathon-2018/}}
  {Hackathon du Département d'Informatique et de Recherche Opérationnelle de l'Université de Montréal (2018)}
    {24 et 25 février 2018}
    {DIRO, Université de Montréal}
    {
      \begin{cvitems}
      \item{Développement d'une intelligence artificielle en Java jouant à un jeu similaire à Lode Runner}
      \end{cvitems}
    }
    \cventry
  {\url{http://hackathon.iro.umontreal.ca/2017/}}
  {Hackathon du Département d'Informatique et de Recherche Opérationnelle de l'Université de Montréal (2017)}
    {25 au 26 février 2017}
    {DIRO, Université de Montréal}
    {
      \begin{cvitems}
      \item{Développement d'une intelligence artificielle en Java capable d'affronter les IA adverses dans le jeu TRON}
      \end{cvitems}
    }
\end{cventries}
