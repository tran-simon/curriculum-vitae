%-------------------------
% Resume in Latex
% Author : Simon Tran
% Based off of: https://github.com/jakegut/resume
% License : MIT
%------------------------

\documentclass[letterpaper,11pt]{article}

\usepackage{latexsym}
\usepackage[empty]{fullpage}
\usepackage{titlesec}
\usepackage{marvosym}
\usepackage[usenames,dvipsnames]{color}
\usepackage{verbatim}
\usepackage{enumitem}
\usepackage[hidelinks]{hyperref}
\usepackage{fancyhdr}
\usepackage[english]{babel}
\usepackage{tabularx}
\input{glyphtounicode}

\usepackage[fixed]{fontawesome5}


%----------FONT OPTIONS----------
% sans-serif
% \usepackage[sfdefault]{FiraSans}
% \usepackage[sfdefault]{roboto}
% \usepackage[sfdefault]{noto-sans}
% \usepackage[default]{sourcesanspro}

% serif
% \usepackage{CormorantGaramond}
% \usepackage{charter}


\pagestyle{fancy}
\fancyhf{} % clear all header and footer fields
\fancyfoot{}
\renewcommand{\headrulewidth}{0pt}
\renewcommand{\footrulewidth}{0pt}

% Adjust margins
\addtolength{\oddsidemargin}{-0.5in}
\addtolength{\evensidemargin}{-0.5in}
\addtolength{\textwidth}{1in}
\addtolength{\topmargin}{-.5in}
\addtolength{\textheight}{1.0in}

\urlstyle{same}

\raggedbottom
\raggedright
\setlength{\tabcolsep}{0in}

% Sections formatting
\titleformat{\section}{
  \vspace{-4pt}\scshape\raggedright\large
}{}{0em}{}[\color{black}\titlerule \vspace{-5pt}]

% Ensure that generate pdf is machine readable/ATS parsable
\pdfgentounicode=1

%-------------------------
% Custom commands
\newcommand{\resumeItem}[1]{
  \item\small{
      {#1 \vspace{-2pt}}
  }
}

\newcommand{\resumeSubheading}[4]{
  \vspace{-2pt}\item
  \begin{tabular*}{0.97\textwidth}[t]{l@{\extracolsep{\fill}}r}
    \textbf{#1} & #2 \\
    \textit{\small#3} & \textit{\small #4} \\
  \end{tabular*}\vspace{-7pt}
}

\newcommand{\resumeSubSubheading}[2]{
  \item
  \begin{tabular*}{0.97\textwidth}{l@{\extracolsep{\fill}}r}
    \textit{\small#1} & \textit{\small #2} \\
  \end{tabular*}\vspace{-7pt}
}

\newcommand{\resumeProjectHeading}[3]{
  \item
  \begin{tabular*}{0.97\textwidth}{l@{\extracolsep{\fill}}r}
    \small#1 & #2 \\
    \textit{\small#3} \\
  \end{tabular*}\vspace{-7pt}
}

\newcommand{\resumeSubItem}[1]{\resumeItem{#1}\vspace{-4pt}}

\renewcommand\labelitemii{$\vcenter{\hbox{\tiny$\bullet$}}$}

\newcommand{\resumeSubHeadingListStart}{\begin{itemize}[leftmargin=0.15in, label={}]}
\newcommand{\resumeSubHeadingListEnd}{\end{itemize}}
\newcommand{\resumeItemListStart}{\begin{itemize}}
\newcommand{\resumeItemListEnd}{\end{itemize}\vspace{-5pt}}

\newcommand{\link}[1]{\emph{\url{#1}}}
\newcommand{\github}[1]{\small{\emph{\href{https://#1}{\faGithubSquare #1}}}}

%-------------------------------------------
%%%%%%  RESUME STARTS HERE  %%%%%%%%%%%%%%%%%%%%%%%%%%%%


\begin{document}

%----------HEADING----------
% \begin{tabular*}{\textwidth}{l@{\extracolsep{\fill}}r}
%   \textbf{\href{http://sourabhbajaj.com/}{\Large Sourabh Bajaj}} & Email : \href{mailto:sourabh@sourabhbajaj.com}{sourabh@sourabhbajaj.com}\\
%   \href{http://sourabhbajaj.com/}{http://www.sourabhbajaj.com} & Mobile : +1-123-456-7890 \\
% \end{tabular*}

\begin{center}
\textbf{\Huge \scshape Simon Tran} \\ \vspace{1pt}
\small \faPhone \href{tel:\phone}{\phone} $|$ \href{mailto:transimon99@gmail.com}{\faEnvelope \emph{transimon99@gmail.com}} $|$
\href{https://linkedin.com/in/transimon99}{\faLinkedin \emph{linkedin.com/in/transimon99}} $|$
\href{https://github.com/tran-simon}{\faGithubSquare\ \emph{github.com/tran-simon}}
\end{center}


%-----------EXPERIENCE-----------
\section{\lang{Experience}{Expérience}}
\resumeSubHeadingListStart

\resumeSubheading
{Amazon}{\lang{May 2022 -- August 2022}{Mai 2022 -- Août 2022}}
{\lang{Software Developer Intern}{Développeur Logiciel Stagiaire}}{Toronto, On}
\resumeItemListStart
\resumeItem{\lang{Researched, designed and implemented a new feature for Amazon's customer service web application}{Recherché, conçu et implémenté une nouvelle fonctionnalité dans une application web pour une équipe de service client d'Amazon}}
\resumeItem{\lang{Created a design document explaining the different requirements and the steps to complete the project}{Créé un document de conception expliquant les différents requis ainsi que les étapes à accomplir pour la complétion du projet}}
\resumeItem{\lang{Designed a GraphQL schema}{Conçu un schéma GraphQL}}
\resumeItem{\lang{Implemented the backend in Java using proprietary Amazon frameworks and tools}{Implémenté le backend en Java en utilisant des outils propriétaires à Amazon}}
\resumeItem{\lang{Modified the frontend to add the new feature in Typescript using React}{Modifié le frontend pour ajouter la nouvelle fonctionnalité en utilisant Typescript et React}}
\resumeItemListEnd

\resumeSubheading
{Google}{\lang{January 2022 -- April 2022}{Janvier 2022 -- Avril 2022}}
{\lang{Software Developer Intern}{Développeur Logiciel Stagiaire}}{\lang{Remote}{À distance}}
\resumeItemListStart
\resumeItem{\lang{Contributed to Fuchsia, Google's open-source operating system, with the Component Framework team. Worked on the "Structured Configuration" Fuchsia project}{Contribué à "Fuchsia", un système d'exploitation à source ouverte de Google, avec l'équipe "Component Framework", sur le projet "Structured Configuration"}}
\resumeItem{\lang{Learned Rust and used C++}{J'ai appris le Rust et utilisé le C++}}
\resumeItem{\lang{Developed "ffx plugins" in Rust, command line tools used to debug Fuchsia components}{J'ai développé des plugins "ffx" en Rust, des outils en ligne de commande utiles pour déverminer des composantes de Fuchsia}}
\resumeItem{\lang{Code review using Gerrit}{Revue de code avec Gerrit}}
\resumeItemListEnd
{\github{www.github.com/tran-simon/fuchsia}}

\resumeSubheading
{Vélocité Conseil}{\lang{September 2020 -- January 2022}{Septembre 2020 -- Janvier 2022}}
{\lang{Fullstack Lead Developer}{Développeur Principal Fullstack}}{Montréal, Qc}
\resumeItemListStart
\resumeItem{\lang{Lead developer of a React app using Typescript and Firebase}{Développeur principal d'une application React avec Typescript et Firebase}}
\resumeItem{\lang{Responsible for key architectural and technological decisions}{Responsable de plusieurs décisions architecturales et technologiques clefs}}
\resumeItem{\lang{Designed parts of the UI/UX using Material-UI}{Conception d'une partie de l'interface utilisateur et de l'expérience utilisateur avec Material-UI}}
\resumeItem{\lang{Responsible for structuring the database in Firebase Realtime Database and Firebase Storage}{Responsable de la structure de la base de données dans Firebase Realtime Database et Firebase Storage}}
\resumeItem{\lang{In charge of application security and user data privacy}{En charge de la sécurité de l'application et de la protection des données des utilisateurs}}
\resumeItemListEnd

\resumeSubheading
{\lang{National Bank of Canada}{Banque Nationale du Canada}}{\lang{May 2020 -- August 2020}{Mai 2020 -- Août 2020}}
{\lang{Intern Developer, Digital Studio}{Développeur Stagiaire, Studio Numérique}}{Montréal, Qc}
\resumeItemListStart
\resumeItem{\lang{Backend (Java), frontend (React, Javascript, html/css) and mobile development (React Native)}{Développement backend (Java), frontend (React, Javascript, html/css) et mobile (React Native)}}
\resumeItem{\lang{Mobile development using React Native}{Développement mobile avec React Native}}
\resumeItemListEnd

\resumeSubheading
{\lang{National Bank of Canada}{Banque Nationale du Canada}}{\lang{May 2019 -- May 2020}{Mai 2019 -- Mai 2020}}
{\lang{Intern Developer, Wealth Management}{Développeur Stagiaire, Gestion de Patrimoine}}{Montréal, Qc}
\resumeItemListStart
\resumeItem{\lang{Backend development in Java using Spring, Springboot, Apache CXF, Swagger, JUnit, Mockito}{Développement backend en Java à l'aide de Spring, SpringBoot, Apache CXF, Swagger, JUnit, Mockito}}
\resumeItem{\lang{Frontend development with React, Javascript, html/css}{Développement FrontEnd avec React, Javascript, html/css}}
\resumeItemListEnd

\resumeSubheading
{\lang{National Bank of Canada}{Banque Nationale du Canada}}{\lang{July 2018 -- August 2018}{Juillet 2018 -- Août 2018}}
{\lang{Intern Developer, Banking Transaction Assets}{Développeur Stagiaire, Actifs Transactionnels Bancaires}}{Montréal, Qc}
\resumeItemListStart
\item {\lang{COBOL development}{Développement en COBOL}}
\resumeItemListEnd

\resumeSubHeadingListEnd


%-----------EDUCATION-----------
\section{\lang{Education}{Éducation}}
\resumeSubHeadingListStart
\resumeSubheading
{Polytechnique Montréal}{Montréal, Qc}
{\lang{Baccalaureate in Computer Engineering (fifth year ongoing)}{Baccalauréat en Génie Informatique (en cours, cinquième année)}}{\lang{August}{Août} 2018 -- \lang{May}{Mai} 2023}
\resumeItemListStart
\resumeItem{\lang{Computer Security and Mobility Concentration}{Concentration Sécurité et Mobilité Informatique}}
\resumeItemListEnd
\resumeSubHeadingListEnd


%-----------PROJECTS-----------
\section{\lang{Projects}{Projets}}
\resumeSubHeadingListStart
\resumeProjectHeading
{\textbf{Jami-Web} $|$ \emph{React, Typescript, Express, WebRTC, Swig, Docker}}{\lang{Fall 2022}{Automne 2022}}
{\lang{Final project at Polytechnique in association with Savoir-faire Linux}{Projet final à Polytechnique en association avec Savoir-faire Linux}}
\resumeItemListStart
\resumeItem{\lang{Creation of the web version of Jami: An open-source distributed video conference application}{Création de la version web de l'application libre distribuée de vidéo conférence Jami}}
\resumeItem{\lang{Frontend in Typescript with React}{Frontend en Typescript avec React}}
\resumeItem{\lang{Implementation of WebRTC for peer-to-peer audiovisual communication}{Implémentation de WebRTC pour la communication audiovisuelle pair-à-pair}}
\resumeItem{\lang{Backend with Express.js in Typescript. Authentication management, REST API and WebSocket connection with the client}{Backend en Express.js avec Typescript. Gestion de l'authentification, API REST et connexion avec le client par WebSocket}}
\resumeItem{\lang{CI using Jenkins}{Intégration continue (CI) avec Jenkins}}
\resumeItem{\lang{Code review using Gerrit}{Revue de code avec Gerrit}}
\resumeItemListEnd
{\github{www.github.com/tran-simon/jami-web}}

\resumeProjectHeading
{\textbf{Babel Reader} $|$ \emph{React, Firebase, GitHub Actions}}{2020}
{\lang{Ebook reader app in React with easy one click translations}{Application en React permettant de lire des livres électroniques dans plusieurs langues}}
\resumeItemListStart
\resumeItem{\lang{Firebase for the hosting and the storage of user data}{Firebase pour l'hébergement et les donnés des utilisateurs}}
\resumeItem{\lang{Continuous integration and deployment using GitHub Actions}{Intégration continue et déploiement continue avec les GitHub Actions}}
\resumeItem{\lang{Demo}{Démo}: \link{https://babel-reader-web.web.app/}}
\resumeItemListEnd
{\github{www.github.com/Babel-Reader/babel-reader-web}}

\resumeProjectHeading
{\textbf{\lang{Crazyflie Drone Exploration}{Drones Explorateurs Crazyflies}} $|$ \emph{C, Python, Flask, React, Typescript, Firebase}}{\lang{Winter 2021}{Hiver 2021}}
{\lang{Third year of university final project}{Projet intégrateur de troisième année d'université}}
\resumeItemListStart
\resumeItem{\lang{Development of an artificial intelligence for a Crazyflie drone in C}{Programmation de l'intelligence artificielle d'un drone Crazyflie en C}}
\resumeItem{\lang{Base station server in python to communicate with the robots}{Serveur station au sol en python pour la communication avec les robots}}
\resumeItem{\lang{Web UI in React and Typescript. Usage of Firebase Realtime Database}{Interface web en React à l'aide de Typescript. Utilisation de Firebase RealtimeDatabase}}
\resumeItem{\lang{Frontend deployed on Firebase}{Interface web déployée sur Firebase}: \link{https://inf3995-100.web.app/}}
\resumeItemListEnd
{\github{www.github.com/tran-simon/inf3995-main}}

\resumeProjectHeading
{\textbf{Polydessin} $|$ \emph{Angular, Typescript, Node, Express, Bitbucket pipelines, Firebase hosting}}{\lang{Winter 2020}{Hiver 2020}}
{{\lang{Second Year of university final project }{Projet intégrateur de deuxième année d'université}}}
\resumeItemListStart
\resumeItem{\lang{Creation of a webapp in Typescript using Angular}{Création d'une webapp en Typescript à l'aide d'Angular}}
\resumeItem{\lang{Backend created using Node and Express}{Création d'un backend avec Node et Express}}
\resumeItem{\lang{Continuous integration and deployment using Bitbucket pipelines}{Intégration continue et déploiement continue avec les pipelines Bitbucket}}
\resumeItem{\lang{Demo version deployed using Firebase}{Déploiement de la version finale à l'aide de Firebase}: \link{https://log2990-104.web.app/}}
\resumeItemListEnd
{\github{www.github.com/tran-simon/LOG2990-104}}

\resumeProjectHeading
{\textbf{Math by Heart} $|$ \emph{Java, Android, MathJax}}{2017}
{\lang{Personnal project of an android open-source app to ease the learning of math formulas}{Application Android open-source pour l'apprentissage de formules mathématiques}}
\resumeItemListStart
\resumeItem{\lang{Usage of Android Studio and the MathJax library}{Utilisation d'Android Studio et de la librairie MathJax (LaTeX) pour afficher les formules mathématiques}}
\resumeItemListEnd
{\github{www.github.com/tran-simon/MathByHeart}}

\resumeProjectHeading
{\textbf{\lang{The Mass Spectrometer}{Le spectromètre de masse}} $|$ \emph{Java, Swing}}{2018}
{\lang{End of CEGEP project: Scientific application in Java to simulate a mass spectrometer}{Projet Intégrateur de fin de CÉGEP: Application scientifique en Java pour simuler un spectromètre de masse}}
\resumeItemListStart
\resumeItem{\lang{Simulation a mass spectrometer, a cyclotron and the behavior of particles under the effect of electromagnetic fields}{Simulation d'un spectromètre de masse, d'un cyclotron et du comportement des particules sous l'effet de divers champs électromagnétiques}}
\resumeItem{\lang{User interface using Java Swing}{Interface usager avec Java Swing}}
\resumeItemListEnd
{\github{www.github.com/tran-simon/Spectrometre}}

\resumeSubHeadingListEnd



%
%-----------PROGRAMMING SKILLS-----------
\section{\lang{Technical Skills}{Habiletés techniques}}
\begin{itemize}[leftmargin=0.15in, label={}]
\small{\item{
\textbf{Languages}{: Javascript, Typescript, Java/Kotlin, Rust, Python, C, C++, HTML/CSS} \\
\textbf{Frameworks}{: React, Angular, Node.js, Express, Flask, Material-UI, Spring/Spring Boot, React Native} \\
\textbf{\lang{Cloud Technologies}{Technologies Infonuagiques}}{: AWS, Google Cloud Platform, Firebas, Azure, Docker, docker-compose} \\
\textbf{\lang{Mobile Development}{Développement mobile}}{: iOS/Android, React Native, Java/Kotlin} \\
\textbf{\lang{Operating System Development}{Développement de systèmes d'exploitation}}{: Fuchsia (Rust, C++)} \\
\textbf{CI/CD}{: Jenkins, GitHub Actions, Bitbucket Pipeline, GitLab Pipelines} \\
\textbf{\lang{Miscellaneous}{Divers}}{: Git, GitHub/Gitlab/Bitbucket/Gerrit, Jira/Trello, Confluence, Windows, macOS, Linux (Arch)} \\
\textbf{\lang{Languages}{Langues}}{: \lang{French, English, Spanish (Beginner)}{Français, Anglais, Espagnol (Débutant)}}
}}
\end{itemize}


%-------------------------------------------
\end{document}
